%% NOTES:

%REMOVE TOO HIGH LR! COLORS ARE REVERSE

%\begin{Imports and headers} 
%%%%%%%%%%%%%%%%%%%%%%%%%%%%%%%%%%%%%%%%%%%%%%%%%%%%%
\documentclass[a4paper]{article}  %Or use titlepage

%% Language and font encodings
\usepackage[english]{babel}
\usepackage[utf8x]{inputenc}
\usepackage[T1]{fontenc}
\usepackage{float} %makes you be able to use H option, places fig there and only there. VERY useful

%% Sets page size and margins
\usepackage[a4paper,top=3cm,bottom=2cm,left=3cm,right=3cm,marginparwidth=1.75cm]{geometry}

%% Useful packages
\usepackage{amsmath}
\usepackage{graphicx}
\usepackage[colorinlistoftodos]{todonotes}
\usepackage[colorlinks=true, allcolors=blue]{hyperref}

%% Header, foot and page number in bottom right corner
\usepackage{fancyhdr}
\usepackage{lipsum}

%% For long comments
\usepackage{comment} 

% Turn on the style
\pagestyle{fancy}
% Clear the header and footer
\fancyhead{}
\fancyfoot{}
% Set the right side of the footer to be the page number
\fancyfoot[R]{\thepage}
\fancyfoot[L]{\date{\today}}
\fancyhead[C]{DD2404 - Signal peptide prediction}

%\end{Imports and headers}

\begin{document}

%\begin{Title and abstract} 
%%%%%%%%%%%%%%%%%%%%%%%%%%%%%%%%%%%%%%%%%%%%%%%%%%%%
\title{\vspace{60mm} \bf 
Signal peptide prediction}  %Fix to center title vertically
\author{Timothy Bergström}
\maketitle
\thispagestyle{fancy}  %Page number for title (\maketitle clears foot and head, so it needs to be after it)
\newpage
\tableofcontents  %OBSOBS, you need to compile the document TWO TIMES for this to work. 
\newpage

\begin{abstract}
Detecting signal peptides (SPs) in proteins computationally is \dots
A neural network model was created and trained with X amount of data, which yielded a model with a classification accuracy of 92\% of the validation set.
The model was used to predict the total amount of SPs in the X and Y organism, which yielded Z and Q SPs respectively, showing that the model is \dots

\lipsum[1]  % Fill with nonsense
\end{abstract}
%\end{Imports and headers} 

%\begin{Introduction} 
%%%%%%%%%%%%%%%%%%%%%%%%%%%%%%%%%%%%%%%%%%%%%%%%%%%%%

\section{Introduction}

Course bla. 

The goal is to create a classifier that can accurately classify proteins with \href{https://en.wikipedia.org/wiki/Signal_peptide}{signal peptides} (SPs) in an organisms \href{https://en.wikipedia.org/wiki/Proteome}{proteome}. 

SPs are short regions in peptides that are on average between 16 to 30 amino acids long, but can be as short as 8 and as long as 65 amino acids \cite{sp_length}. SPs function is to translocate proteins to different parts of the cell, such as insertion into membranes, moving proteins to organelles or to help the cell excrete the proteins \cite{sp_wiki}. 

\subsection{The structure of a signal peptide}

An SP is always located near the \href{https://en.wikipedia.org/wiki/N-terminus}{N-terminal} of a protein and have three regions; the n-region, the h-region and the c-region.

The n-region is the first region, which usually contains positively charged amino acids, making the region very polar and hydrophilic. Then comes the h-region, which contains mostly hydrophobic amino acids that forms an \href{https://en.wikipedia.org/wiki/Alpha_helix}{$\alpha$-helix} and lastly the c-region, which is usually a sequence of 3 to 8 non-charged polar amino acids. After the c-region, there is usually a cleavage site that have a tendency to be surrounded by small, uncharged amino acids.

When analyzing an organisms proteome, the SPs can give an indication where in the cell the proteins are translocated to which could be of an interest when researching an organism. Not all proteins contain SPs, which is why signal peptide prediction is useful for determining the layout of the cell and for annotating organisms proteomes. The problem with signal peptide prediction is that polar regions in trans-membrane proteins can mistakenly be predicted as the h-region of an SP, which makes predicting SPs in trans-membrane proteins difficult.

%\end{Introduction} 

%\begin{Methods and models}
%%%%%%%%%%%%%%%%%%%%%%%%%%%%%%%%%%%%%%%%%%%%%%%%%%%%%
\section{Methods and models}

\subsection{Design choices}
Design choices, python, keras etc.


\subsection{Data set}
Scraped from websites, extracted etc. POS and NEG, see Table~\ref{tab:datatable}

\begin{table}[H]
\centering
\begin{tabular}{l | c | r} %If you add more columns, change here
Type & Positive & Negative \\\hline
TM & 10 & 30 \\
non-TM & 1 & 2 \\
Unknown & a & b
\end{tabular}
\caption{\label{tab:datatable}Data table.}
\end{table}

Typical signal peptide from dataset:
\begin{figure}[H]
\centering
\includegraphics[width=0.7\textwidth]{pictures/seqlogo.png}
\caption{\label{fig:seqlogo}Extracted data from all positive samples}
\end{figure}

How you handled data, noise, ignore etc for bad samples. Seq files \verb|.fasta| file containing sequences

 See Figure~\ref{fig:seqlogo}

\subsection{Neural Network structure}
LSTM, CNN and dropouts. Reference self to text generator?

Model picture and maybe summary?


\subsection{Related works}

There are multiple softwares and websites that predicts signal peptides, such as SignalP, SignalBlast and Predisi to name a few \cite{sp_predict1} \cite{sp_predict2} \cite{sp_predict3}. Neural Networks (NN) are the most used methods while some sites use Hidden Markov Models (HMM) instead. Similar methods are used to predict other biological problems, such as motifs and trans-membrane regions in proteins.

%\end{Methods and models}

%\begin{Evaluating model}
%%%%%%%%%%%%%%%%%%%%%%%%%%%%%%%%%%%%%%%%%%%%%%%%%%%%%
\section{Evaluating model}

\subsection{Model performance}
ROC-curve, CM-matrix for full data full data.

\subsection{Signal peptide detection in transmembrane proteins}
Evaluate transmembrane and non transmembrane acc/loss.
CM for TM and non-TM

\subsection{Signal peptide detection in proteomes}
Test two organisms, correct for CM

\subsection{Generative model}
A generative model was created that can create SPs. See in appendix \ref{GAN}.

%\end{Evaluating model}

%\begin{Conclusion}
%%%%%%%%%%%%%%%%%%%%%%%%%%%%%%%%%%%%%%%%%%%%%%%%%%%%%
\section{Conclusion}

Is it good enough? Can the model be improved? Can you use other methods (ex HMM, CNN, grouping etc...)

%\end{Conclusion}

%\begin{References}
%%%%%%%%%%%%%%%%%%%%%%%%%%%%%%%%%%%%%%%%%%%%%%%%%%%%%
\newpage
%% References
\bibliographystyle{alpha}
\begin{thebibliography}{9}
\bibitem{mrfool} Mrfool {\em Readings compiled for History 21.479.} 1991.
\bibitem{sp_length} Henrik Nielsen \url{http://www.cbs.dtu.dk/services/SignalP-1.1/sp_lengths.html}
\bibitem{sp_wiki} Signal peptides wikipedia \url{https://en.wikipedia.org/wiki/Signal_peptide}
\bibitem{sp_predict1} Henrik Nielsen, {\em Predicting Secretory Proteins with SignalP}, In Kihara, D (ed): Protein Function Prediction (Methods in Molecular Biology vol. 1611),  pp. 59-73, Springer 2017, doi: 10.1007/978-1-4939-7015-5\_6, PMID: 28451972, \url{http://www.cbs.dtu.dk/services/SignalP/}  %CHAR _ makes Latex crash, use \_
\bibitem{sp_predict2} Karl Frank; Manfred J. Sippl, {\em High Performance Signal Peptide Prediction Based on Sequence Alignment Techniques}, Bioinformatics, 24, pp. 2172-2176 (2008), \url{http://sigpep.services.came.sbg.ac.at/signalblast.html}
\bibitem{sp_predict3}  Karsten Hiller,  {\em PrediSi}, Institute for Microbiology Technical University of Braunschweig,\url{http://www.predisi.de/}
\end{thebibliography}

%\end{References}

%\begin{Appendix}
%%%%%%%%%%%%%%%%%%%%%%%%%%%%%%%%%%%%%%%%%%%%%%%%%%%%%
\newpage

\section{Appendix}

\subsection{Training the model}
Train time, gpu used, specs, time to train etc...
Gpu is needed to improve training speed. Add epoch time for gpu and cpu

Everything that the person reading don't want to know


\subsection{Pitfalls when training a model}
Concept of under and overfitting and how to prevent it. Example pictures of bad models.

Overfitting \dots

other stuff too

more stuff

\begin{figure}[H] %Rank in order
\center
\includegraphics[width=0.7\textwidth]{pictures/overfit.png}
\caption{\label{fig:overfit}Overfitted}
\end{figure}

Too high learning rate \dots

other stuff too

more stuff

\begin{figure}[H]
\center
\includegraphics[width=0.7\textwidth]{pictures/high_lr.png}
\caption{\label{fig:high_lr}Too high learning rate}
\end{figure}


\subsection{Generative Model} \label{GAN}


%\end{Appendix}

\end{document}


%\begin{DUMP}

\begin{comment}

\LaTeX{} is great at great tool. Let $X_1, X_2, \ldots, X_n$ be a sequence of independent and identically distributed random variables with $\text{E}[X_i] = \mu$ and $\text{Var}[X_i] = \sigma^2 < \infty$, and let
\[S_n = \frac{X_1 + X_2 + \cdots + X_n}{n} = \frac{1}{n}\sum_{i}^{n} X_i\]
 $n$ approaches infinity, the random variables $\sqrt{n}(S_n - \mu)$ converge in distribution to a normal $\mathcal{N}(0, \sigma^2)$.

List of things \dots
\begin{enumerate}
\item Problems,
\item Structure data
\end{enumerate}
\dots More things \dots
\begin{itemize}
\item Things to add,
\item Testing
\end{itemize}

\end{comment}

%\end{DUMP}
